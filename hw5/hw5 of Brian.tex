\documentclass{article}

\usepackage{fancyhdr}
\usepackage{extramarks}
\usepackage{amsmath}
\usepackage{amsthm}
\usepackage{amsfonts}
\usepackage{tikz}
\usepackage[plain]{algorithm}
\usepackage{algpseudocode}
\graphicspath{{img/}}
\usetikzlibrary{automata,positioning}

%
% Homework Details
%   - Title
%   - Class
%   - Author
%   - UID
%

\newcommand{\hmwkTitle}{Homework\ \#5}
\newcommand{\hmwkClass}{Com Sci 146}
\newcommand{\hmwkAuthorName}{\textbf{Jingyue Shen}}
\newcommand{\UID}{\textbf{704797256}}


%
% Basic Document Settings
%

\topmargin=-0.45in
\evensidemargin=0in
\oddsidemargin=0in
\textwidth=6.5in
\textheight=9.0in
\headsep=0.25in

\linespread{1.1}

\pagestyle{fancy}
\lhead{\hmwkAuthorName}
\chead{\hmwkClass\ : \hmwkTitle}
\rhead{\firstxmark}
\lfoot{\lastxmark}
\cfoot{\thepage}

\renewcommand\headrulewidth{0.4pt}
\renewcommand\footrulewidth{0.4pt}

\setlength\parindent{0pt}

%
% Create Problem Sections
%

\newcommand{\enterProblemHeader}[1]{
    \nobreak\extramarks{}{Problem \arabic{#1} continued on next page\ldots}\nobreak{}
    \nobreak\extramarks{Problem \arabic{#1} (continued)}{Problem \arabic{#1} continued on next page\ldots}\nobreak{}
}

\newcommand{\exitProblemHeader}[1]{
    \nobreak\extramarks{Problem \arabic{#1} (continued)}{Problem \arabic{#1} continued on next page\ldots}\nobreak{}
    \stepcounter{#1}
    \nobreak\extramarks{Problem \arabic{#1}}{}\nobreak{}
}

\setcounter{secnumdepth}{0}
\newcounter{partCounter}
\newcounter{homeworkProblemCounter}
\setcounter{homeworkProblemCounter}{1}
\nobreak\extramarks{Problem \arabic{homeworkProblemCounter}}{}\nobreak{}

%
% Homework Problem Environment
%
% This environment takes an optional argument. When given, it will adjust the
% problem counter. This is useful for when the problems given for your
% assignment aren't sequential. See the last 3 problems of this template for an
% example.
%
\newenvironment{hw}[1][-1]{
    \ifnum#1>0
        \setcounter{homeworkProblemCounter}{#1}
    \fi
    \section{Problem \arabic{homeworkProblemCounter}}
    \setcounter{partCounter}{1}
    \enterProblemHeader{homeworkProblemCounter}
}{
    \exitProblemHeader{homeworkProblemCounter}
}

%
% Title Page
%

\title{
    \textmd{\textbf{\hmwkClass:\ \hmwkTitle}}\\
    \author{\hmwkAuthorName}
    \UID
}

\date{}

\renewcommand{\part}[1]{\textbf{\large Part \Alph{partCounter}}\stepcounter{partCounter}\\}

%
% Various Helper Commands
%

% Useful for algorithms
\newcommand{\alg}[1]{\textsc{\bfseries \footnotesize #1}}

% For derivatives
\newcommand{\deriv}[1]{\frac{\mathrm{d}}{\mathrm{d}x} (#1)}

% For partial derivatives
\newcommand{\pderiv}[2]{\frac{\partial}{\partial #1} (#2)}

% Integral dx
\newcommand{\dx}{\mathrm{d}x}

% Alias for the Solution section header
\newcommand{\solution}{\textbf{\large Solution}}


% Probability commands: Expectation, Variance, Covariance, Bias
\newcommand{\E}{\mathrm{E}}
\newcommand{\Var}{\mathrm{Var}}
\newcommand{\Cov}{\mathrm{Cov}}
\newcommand{\Bias}{\mathrm{Bias}}

\begin{document}

\maketitle

%CH 1: 5 8
\begin{hw}[1]
  \part{}
  We only model the word frequency in each class, but ignore the relation between current word and the words before and after it.\\
  \part{}
  $log^{Pr(D_i,y_i)} = log^{Pr(D_i|y_i)}+log^{Pr(y_i)} = log^{\frac{n!}{a_i!b_i!c_i!}\alpha_0^{a_i(1-y_i)}\alpha_1^{a_iy_i}\beta_0^{b_i(1-y_i)}\beta_1^{b_iy_i}\gamma_0^{c_i(1-y_i)}\gamma_1^{c_iy_i}}+ log^{\theta^{y_i}(1-\theta)^{1-y_i}}$\\
  \part{}
  $\frac{\partial{log^{\prod_{i=1}^m{P(D_i,y_i)}}}}{\partial{\alpha_1}} = 0$\\\\
  $=> \frac{\partial{\sum_{i=1}^m{a_iy_ilog^{\alpha_1}+c_iy_ilog^{1-\alpha_1-\beta_1}}}}{\partial{\alpha_1}} = 0$\\\\
  $=> \frac{\sum_{i=1}^m{a_iy_i}}{\alpha_1} - \frac{\sum_{i=1}^m{c_iy_i}}{1-\alpha_1-\beta_1} = 0$\\\\
  $=> \alpha_1(\sum{a_iy_i}+\sum{c_iy_i}) = (1-\beta_1)\sum{a_iy_i}$        (1)\\\\
  
  Take partial derivative with respect to $\beta_1$, we can get \\
  $\beta_1(\sum{b_iy_i}+\sum{c_iy_i}) = (1-\alpha_1)\sum{b_iy_i}$        (2)\\
  
  Solve (1) and (2) together, we can get $\alpha_1 = \frac{\sum{a_iy_i}}{\sum{a_iy_i}+\sum{b_iy_i}+\sum{c_iy_i}}$, $\beta_1 = \frac{\sum{b_iy_i}}{\sum{a_iy_i}+\sum{b_iy_i}+\sum{c_iy_i}}$\\
  Follow the same procedure, we can get $\gamma_1 = \frac{\sum{c_iy_i}}{\sum{a_iy_i}+\sum{b_iy_i}+\sum{c_iy_i}}$, and $\alpha_0 = \frac{\sum{a_i(1-y_i)}}{\sum{a_i(1-y_i)}+\sum{b_i(1-y_i)}+\sum{c_i(1-y_i)}}$ ,  $\beta_0 = \frac{\sum{b_i(1-y_i)}}{\sum{a_i(1-y_i)}+\sum{b_i(1-y_i)}+\sum{c_i(1-y_i)}}$, $\gamma_0 = \frac{\sum{c_i(1-y_i)}}{\sum{a_i(1-y_i)}+\sum{b_i(1-y_i)}+\sum{c_i(1-y_i)}}$
  
\end{hw}
\begin{hw}[2]
  \part{}
  The missing transition probabilities are $P(q_{t+1} = 1| q_t = 2) = 1-q_{11} = 0$ and $P(q_{t+1} = 2 | q_t = 2) = 1-q_{12} = 0$, and the two missing probabilities are $e_1(B) = P(O_t = B|q_t=1) =1-e_1(A) = 0.01$ and $e_2(A) = P(O_t = A | q_t = 2) = 1-e_2(B)= 0.49$\\
  \part{}
  $P(generate A) = \pi_1*e_1(A) + \pi_2*e_2(A) = 0.49\times 0.99 + 0.51 \times 0.49 = 0.735$
  $P(genearte B) = \pi_1*e_1(B) + \pi_2*e_2(B) = 0.49\times 0.01 + 0.51 \times 0.51 = 0.265$\\
  So A is the most symbnol appear in the first position of the sequence.\\
  \part{}
  From the above we know that the most probable first symbol is A. Since $q_{11}=q_{12} = 1$, no matter the first state is 1 or 2, the second and third states must be 1.
  since $e_1(A) = 0.99$, so the most probable symbol in second and third positions are As. So the sequence of three output symbols that has the highest probability being generated is A A A.
  
  
\end{hw}
\begin{hw}[3]
  \part{}
  It is a bad idea since if k is not fixed, $J(c,\mu,k)$ can reach zero by having n centroids whose positions are the n datapoints. Under this case, k =n, $\mu_i = x^{(i)}$,$c^{(i)} = i$\\
  Part (d)\\
  \includegraphics[width=6cm,height=6cm]{cen_iter1}\\
  Center:[ 1.53574549,  0.70317209], [ 0.04416798,0.548694  ], [ 0.05848729,  0.35555227]\\
  \includegraphics[width=6cm,height=6cm]{cen_iter2}\\
  Center: [ 1.55846957,  0.69620803], [ 0.08399041, 0.58882485], [ 0.06765327,  0.36774192] \\
  \includegraphics[width=6cm,height=6cm]{cen_iter3}\\
  Center:[ 1.57957558,  0.6886763 ], [ 0.14104785,0.64172345],[ 0.06969682,  0.38783907]  \\
  \includegraphics[width=6cm,height=6cm]{cen_iter4}\\
  Center:[ 1.62366096,  0.65422844],[ 0.32109265, 0.82675015],[ 0.05362096,  0.42350331]\\
  \includegraphics[width=6cm,height=6cm]{cen_iter5}\\
  Center:[ 1.66538394,  0.63481783], [ 0.69870477, 1.04031048], [ 0.04733828,  0.46751536]\\
  \includegraphics[width=6cm,height=6cm]{cen_iter6}\\
  Center: [ 1.93884659,  0.48088921],[ 0.99037342, 1.00390775]),[ 0.04917974,  0.4810944 ]\\
  \includegraphics[width=6cm,height=6cm]{cen_iter7}\\
  Center:  [ 2.00594139,  0.47723895],[ 1.01605529, 0.95288767], [ 0.04917974,  0.4810944 ]\\
  \includegraphics[width=6cm,height=6cm]{cen_iter8}\\
  Center:[ 2.03085592,  0.46538378], [ 1.04063507, 0.9409604 ],[ 0.04917974,  0.4810944 ]\\
  \includegraphics[width=6cm,height=6cm]{cen_iter9}\\
  Center:[ 2.03085592,  0.46538378], [ 1.04063507,0.9409604 ],[ 0.04917974,  0.4810944 ]\\
  \pagebreak
  
  \textbf{Part E}\\\\
  \includegraphics[width=6cm,height=6cm]{me_iter1}\\
  Center:[ 1.50765091,  0.71434218],[ 0.04054534,  0.52890919],[ 0.06755541,  0.31829728]\\
  \includegraphics[width=6cm,height=6cm]{me_iter2}\\
  Center:[ 1.50765091,  0.71434218],[ 0.04054534,  0.52890919],[ 0.06755541,  0.31829728]\\
  \includegraphics[width=6cm,height=6cm]{me_iter3}\\
  Center: [ 1.50765091,  0.71434218], [ 0.04054534,  0.52890919],[ 0.06755541,  0.31829728]\\
\end{hw}
  
\pagebreak
\end{document}