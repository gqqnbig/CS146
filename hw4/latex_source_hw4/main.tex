\documentclass[11pt]{article}
\usepackage{course}

\begin{document}

\ctitle{4}{Boosting, Multi-class Classification}{March 8, 2018, 11:59pm}
\author{}
\date{}
\maketitle
\vspace{-0.75in}

\vspace{-11pt}
% \blfootnote{Parts of this assignment are adapted from course material by Jenna Wiens (UMich) and Tommi Jaakola (MIT).}

\ifsoln
\else
\section*{Submission instructions}
\begin{itemize}
\item 
Submit your solutions electronically on the course Gradescope site as PDF files.
\item If you plan to typeset your solutions, please use the LaTeX solution template. If you must submit scanned handwritten solutions, please use a black pen on blank white paper and a high-quality scanner app.

\end{itemize}
\fi


\section {Boosting - 40 points}
  Consider the following examples $(x,y) \in \mathbb{R}^2$ ({\em i} is the example index):
  \begin{center}
    \begin{tabular}{|c|c|c|c|}
      \hline
      {\em i}  & $x$  & $y$ & Label \\
      \hline
      {\em 1}  & 0  & 8 & $-$ \\
      \hline
      {\em 2}  & 1  & 4 & $-$ \\
      \hline
      {\em 3}  & 3  & 7 & $+$ \\
      \hline
      {\em 4}  & -2  & 1 & $-$ \\
      \hline
      {\em 5}  & -1  & 13 & $-$ \\
      \hline
      {\em 6}  & 9  & 11 & $-$ \\
      \hline
      {\em 7}  & 12 & 7 & $+$ \\
      \hline
      {\em 8}  & -7  & -1 & $-$ \\
      \hline
      {\em 9}  & -3  & 12 & $+$ \\
      \hline
      {\em 10} & 5  & 9 & $+$ \\
      \hline
    \end{tabular}
  \end{center}
    % {\bf Add indices to the rows of both tables?}

    \begin{table}[!t]
      {\centering
        \begin{tabular}{|c|c||c|c|c|c||c|c|c|c|}

          \hline
          & & \multicolumn{4}{c||}{Hypothesis 1 (1st iteration)}
	  & \multicolumn{4}{c|}{Hypothesis 2 (2nd iteration)} \\
          \cline{3-10}
          {\em i} & Label & $D_0$ & $f_1 \equiv $ & $f_2 \equiv $ & $h_1\equiv$ & $D_1$ &  $f_1 \equiv $ & $f_2 \equiv $ & $h_2 \equiv $ \\
          & & & [$x > 2$] & [$y > 6$] & [$x > 2$] & & [$x > 10$] & [$y > 11$] & [$y > 11$] \\

          \tiny{(1)} & \tiny{(2)} & \tiny{(3)} & \tiny{(4)} &  \tiny{(5)} & \tiny{(6)} & \tiny{(7)} & \tiny{(8)} & \tiny{(9)} & \tiny{(10)}\\
          \hline \hline
          {\em 1} & $-$ & 0.1 & $-$ & $+$ & $-$ & $\frac{1}{16}$ & $-$ & $-$ & $-$ \\
          \hline
          {\em 2} & $-$ & 0.1 & $-$ & $-$ & $-$ & $\frac{1}{16}$ & $-$ & $-$ & $-$ \\
          \hline
          {\em 3} & $+$ & 0.1 & $+$ & $+$ & $+$ & $\frac{1}{16}$ & $-$ & $-$ & $-$ \\
          \hline
          {\em 4} & $-$ & 0.1 & $-$ & $-$ & $-$ & $\frac{1}{16}$ & $-$ & $-$ & $-$ \\
          \hline
          {\em 5} & $-$ & 0.1 & $-$ & $+$ & $-$ & $\frac{1}{16}$ & $-$ & $+$ & $+$ \\
          \hline
          {\em 6} & $-$ & 0.1 & $+$ & $+$ & $+$ & $\frac{1}{4}$ & $-$ & $-$ & $-$ \\
          \hline
          {\em 7} & $+$ & 0.1 & $+$ & $+$ & $+$ & $\frac{1}{16}$ & $+$ & $-$ & $-$ \\
          \hline
          {\em 8} & $-$ & 0.1 & $-$ & $-$ & $-$ & $\frac{1}{16}$ & $-$ & $-$ & $-$ \\
          \hline
          {\em 9} & $+$ & 0.1 & $-$ & $+$ & $-$ & $\frac{1}{4}$ & $-$ & $+$ & $+$ \\
          \hline
          {\em 10} & $+$ & 0.1 & $+$ & $+$ & $+$ & $\frac{1}{16}$ & $-$ & $-$ & $-$ \\
          \hline
        \end{tabular}
        \caption{Table for Boosting results}\label{table:ltu}}
    \end{table}


  In this problem, you will use Boosting to learn a hidden Boolean function from this set of examples.
We will use two rounds of AdaBoost to learn a hypothesis for this
    data set. In each round, AdaBoost chooses a weak learner that minimizes the error $\epsilon$. As weak learners, use hypotheses of the form (a)~$f_1 \equiv [x
    > \theta_x]$ or (b)~$f_2 \equiv [y > \theta_y]$, for some integers $\theta_x, \theta_y$ (either one of the two forms, not a disjunction of the two). There should be no need to try many values of $\theta_x, \theta_y$;
    appropriate values should be clear from the data. When using log, use base 2. 


  \begin{enumerate}
  \item {\bf [10 points]}  Start the first round with a uniform distribution $D_0$.  Place the value for $D_0$ for each example in the third column of Table~\ref{table:ltu}. Write the new representation of the data in terms of the {\em rules of thumb}, $f_1$ and $f_2$, in the fourth and fifth columns of Table~\ref{table:ltu}.

  

  \item {\bf [10 points]}
    Find the hypothesis given by the weak learner that minimizes the error
    $\epsilon$ for that distribution.  Place this hypothesis as the heading to the
    sixth column of Table~\ref{table:ltu}, and give its prediction for each example in that column.

  \item {\bf [10 points]} Now compute $D_1$ for each example, find the new best weak learners $f_1$ and $f_2$, and select hypothesis that minimizes error on this distribution, placing these values and predictions in the seventh to tenth columns of Table~\ref{table:ltu}.
  
  $$\alpha_0 = \frac{1}{2}\log_2{\left(\frac{1-\epsilon_0}{\epsilon_0}\right)} = \frac{1}{2}\log_2{\left(\frac{1-\frac{2}{10}}{\frac{2}{10}}\right)} = \frac{1}{2}\log_2{4} = 1$$
  $$\sum_i{D_1(i)} = \sum_i{\frac{D_0(i)}{Z_0}\;2^{-\alpha_0y_ih_t(x_i)}} = \frac{\frac{1}{10}(2^{-1}\times8 + 2\times2)}{Z_0} = 1$$
  $$\text{Hence,}\quad Z_0 = \frac{4}{5}\quad \text{and}\quad D_1(i) = \frac{1}{8}\times2^{-y_ih_t(x_i)}$$


  \item {\bf [10 points]} Write down the final hypothesis produced by AdaBoost.
  $$\epsilon_1 = 4\times\frac{1}{16} = \frac{1}{4}$$
  $$\alpha_1 = \frac{1}{2} \times \log_2{\left(\frac{1-\frac{1}{4}}{\frac{1}{4}}\right)} = \frac{1}{2}\log_2{3}$$
  $$H_{final}(x) = \text{sgn}\left(h_1(x) + \left(\frac{1}{2}\log_2{3}\right)h_2(x)\right)$$

\end{enumerate}

\textbf{What to submit:} Fill out Table~\ref{table:ltu} as explained, show computation of $\alpha$ and $D_1(i)$, and give the final hypothesis, $H_{\textit{final}}$.



\iffalse
\section{AdaBoost \problemworth{30}}

In the lecture on ensemble methods, we said that in iteration $t$, AdaBoost is picking $(h_t, \beta_t)$ that minimizes the objective:
\begin{eqnarray*}
(h_t^*(\vect{x}), \beta_t^*) & = & \argmin_{(h_t(\vect{x}), \beta_t)}\sum_n w_t(n) e^{-y_n \beta_t h_t(\vect{x}_n)} \\ 
& =& \argmin_{(h_t(\vect{x}), \beta_t)}(e^{\beta_t}- e^{-\beta_t})  \sum_n w_t(n)\mathbb{I}[y_n \ne h_t(\vect{x}_n)] \\
&& \qquad\qquad\qquad\qquad\qquad + e^{-\beta_t} \sum_n w_t(n)
\end{eqnarray*}
We define the weighted misclassification error at time t, $\epsilon_t$ to be $\epsilon_t = \sum_n w_t(n)\mathbb{I}[y_n \ne h_t(\vect{x}_n)]$. Also the weights are normalized so that $\sum_n w_t(n)=1$. 

\begin{enumerate}
\item \itemworth{3}
Take the derivative of the above objective function with respect to $\beta_t$ and set it to zero to solve for $\beta_t$ and obtain the update for $\beta_t$.

\solution{
\begin{eqnarray*}
J(\beta) & =& (e^{\beta_t}- e^{-\beta_t})  \epsilon_t  + e^{-\beta_t} 
\end{eqnarray*}
\begin{eqnarray*}
\frac{\partial J(\beta_t)}{\partial \beta_t} & =& (e^{\beta_t}+ e^{-\beta_t})  \epsilon_t  - e^{-\beta_t} \\
&=& 0 
\end{eqnarray*}
Solving gives:
\begin{eqnarray*}
e^{2\beta_t}+ 1  &=& \frac{1}{\epsilon_t} \\
e^{2\beta_t}  &=& \frac{1-\epsilon_t}{\epsilon_t} \\
\beta_t  &=& \frac{1}{2} \log\left(\frac{1-\epsilon_t}{\epsilon_t}\right)
\end{eqnarray*}

%[2pt for computing the derivative correctly; 1 pt for answer]
}

\item \itemworth{2} Suppose the training set is linearly separable, and we use a hard-margin linear support vector machine (no slack) as a base classifier. In the first boosting iteration, what would the resulting $\beta_1$ be?

\solution{
The value of $\beta_1$ is infinite. Increasing $\beta_1$ will decrease all the training losses since $y_n h(\vect{x}_n) > 0$ for all $n$.

That is, if the training set is linearly separable and we use a hard-margin SVM with no slack as a base classifier, then a single classifier is sufficient for the ensemble classifier. Thus, in stage $t=1$, $h(\vect{x})$ will correspond to a linear decision boundary that correctly classifies all points ($\epsilon_1 = 0$) so that $\beta_1 = \infty$.

%[1 pt for recognizing single base classifier perfectly separates training set; 1 pt for recognizing $\beta_1 = \infty$]
}
\end{enumerate}
\fi
\section{Multi-class classification - 60 points}

Consider a multi-class classification problem with $k$ class
labels $\{1, 2, \ldots k\}$. Assume that we are given $m$
examples, labeled with one of the $k$ class labels. Assume, for
simplicity, that we have $m/k$ examples of each type.

Assume that you have a learning algorithm $L$ that can be used
to learn Boolean functions. (E.g., think about $L$ as the
Perceptron algorithm). We would like to explore several ways to
develop learning algorithms for the multi-class classification
problem.

There are two schemes to use the algorithm $L$ on the given data set, and produce a multi-class classification:
\begin{itemize}
\item {\bf One vs.~All:} For every label $i \in [1,k]$, a classifier is learned over the following data set: the examples labeled with the label $i$ are considered ``positive'', and examples labeled with any other class $j \in [1,k], j \neq i$ are considered ``negative''.
\item {\bf All vs.~All:} For every pair of labels $\langle i, j \rangle$, a classifier is learned over the following data set: the examples labeled with one class $i \in [1,k]$ are considered ``positive'', and those labeled with the other class $j \in [1,k], j \neq i$ are considered ``negative''.
\end{itemize}
%
\vspace{-3mm}
\begin{enumerate}
\item {\bf [20 points]} For each of these two schemes, answer the following:
\begin{enumerate}
\item How many classifiers do you learn?

One vs. All: $k$, \\times
All vs. All: $\frac{k(k-1)}{2}.$
\item How many examples do you use to learn each classifier within the scheme?

One vs. All: $m$, \\
All vs. All: $2m/k.$
\item How will you decide the final class label (from \{1, 2, \ldots, k\}) for each example?

One vs. All: $y^\text{test} = \argmax_{y_i \in [1, k]}{W_{y_i}^T X^{\text{test}}}$, which means that take the classifer with largest product of all the $k$ weight vectors and feature vector to be predicted. \\
All vs. All: either by majority stategy, where label i wins on $X^{\text{test}}$ more often than any other labels, or by tournament, where start with $m/2$ pairs and continue with winners.
\item What is the computational complexity of the training process?

According to questions on Piazza, the computational complexity here is time complexity.\\
One vs. All: $O\left(mk\right)$, \\
All vs. All: $O\left(\frac{k(k-1)}{2}\times \frac{2m}{k}\right) = O\left(m(k-1)\right) = O\left(mk\right). $
\end{enumerate}
\item {\bf [5 points]} Based on your analysis above of two schemes individually, which scheme would you prefer? Justify.

If the only critierion here is computational time complexity, both schemes are equivalent. If we consider the space complexity, then One vs. All is better.
\item {\bf [5 points]} You could also use a \textsc{KernelPerceptron} for a two-class classification. We could also use the algorithm to learn a multi-class classification. Does using a \textsc{KernelPerceptron} change your analysis above? Specifically, what is the computational complexity of using a \textsc{KernelPerceptron} and which scheme would you prefer when using a \textsc{KernelPerceptron}? 

In kernal perception, $y_j\left(\sum_{1...n}{\alpha_i y_i K(X_i, X_j)}\right)$ is computed for each iteration, where $n$ is number of examples and $\alpha_i$ is the count of examples as constituents in weight vector. As the algorithm proceeds, more $\alpha_i$ will be nonzero, so the computational complexity increases for 1 single classifier, which is bounded by $O\left(d + 2d + ...\:+ nd\right) = O(n^2 d)$, where $d$ is the dimension of feature vectors. For One vs. All, $n = m$; for All vs. All, $n = 2m/k$. Clearly, All vs. All has better computational time complexity of the overall training process.
\item {\bf [10 points]} We are given a magical black-box binary classification algorithm (we don’t know how it works, but it just does!) which has a learning time complexity of O($dn^2$), where $n$ is the total number of training examples supplied (positive+negative) and $d$ is the dimensionality of each example.
What are the overall training time complexities of the all-vs-all and the one-vs-all
paradigms, respectively, and which training paradigm is most efficient?

One vs. All: $O\left(dm^2k\right)$, \\
All vs. All: $O\left(d\left(\frac{2m}{k}\right)^2 \times \frac{k(k-1)}{2}\right) = O(dm^2)$.\\
Hence, as long as $k$ is not a constant, them All vs. All is better.
\item {\bf [10 points]} We are now given another magical black-box binary classification algorithm (wow!) which has a learning time complexity of O($d^2 n$), where $n$ is the total number of training examples supplied (positive+negative) and $d$ is the dimensionality of each example.
What are the overall training time complexities of the all-vs-all and the one-vs-all paradigms, respectively, and which training paradigm is most efficient, when using this new classifier?

One vs. All: $O\left(d^2mk\right)$, \\
All vs. All: $O\left(d^2\times\frac{2m}{k} \times \frac{k(k-1)}{2}\right) = O(d^2 m(k-1)) = O(d^2mk)$.\\
Both schemes are equivalent in terms of time complexity.

\item {\bf [10 points]} Suppose we have learnt an all-vs-all multi-class classifier and now want to proceed to predicting labels on unseen examples.

We have learnt a simple linear classifier with a weight vector of dimensionality $d$ for each of the $ m(m-1)/2$ classes ($w_i^T x = 0$ is the simple linear classifier hyperplane for each  $i =[1, \cdots , m(m-1)/2] )$

We have two evaluation strategies to choose from. For each example, we can:
\begin{itemize}
  \item \textbf{Counting}: Do all predictions then do a majority vote to decide class label
  \item \textbf{Knockout}: Compare two classes at a time, if one loses, never consider it
again. Repeat till only one class remains.
\end{itemize}
What are the overall evaluation time complexities per example for Counting and
Knockout, respectively?

Counting: $O(\frac{k(k-1)}{2}*d) = O(k^2 d)$, \\
Knockout: $O\left(\left(k+\frac{k}{2} + ... + 1\right) \times d\right) = O(kd)$
Knockout is more efficient.
\end{enumerate}


\end{document}